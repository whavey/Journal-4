\documentclass[conference]{IEEEtran}
\IEEEoverridecommandlockouts
% The preceding line is only needed to identify funding in the first footnote. If that is unneeded, please comment it out.
\usepackage{cite}
\usepackage{amsmath,amssymb,amsfonts}
\usepackage{algorithmic}
\usepackage{graphicx}
\usepackage{textcomp}
\usepackage{xcolor}
\usepackage{hyperref}
\hypersetup{
    colorlinks=true,
    linkcolor=blue,
    filecolor=magenta,      
    urlcolor=cyan,
}
\def\BibTeX{{\rm B\kern-.05em{\sc i\kern-.025em b}\kern-.08em
    T\kern-.1667em\lower.7ex\hbox{E}\kern-.125emX}}
\begin{document}

\title{Wayne Havey Journal \#4\\
}

\author{\IEEEauthorblockN{1\textsuperscript{st} Wayne R. Havey III}
\IEEEauthorblockA{\textit{PhD Student: Security Track} \\
\textit{UCCS}\\
Colorado Springs, CO \\
whavey@uccs.edu}
}

\maketitle

\section{This Weeks Learning: Survey Paper}
\subsection{Introduction}
This week has been focused on taking the raw notes to formulate a classification concept and trying to see how the classification holds up. Additionally this week has been as much writing focused as reading focused.
\subsection{Formulating a Classification}
I found that the high level note taking approach from abstracts made it easy to form a classification. The part that Im still trying to figure out is what type of model would best represent the classification. I think there are benefits to using both a multidimensional model and a tree\/hierarchical type model. I wonder how often survey papers include multiple types of models for the same type of classification, or if most classifications are obviously more suited to one or another. What I would really like to attempt is to make an entirely different type of model similar to a logic diagram or a finite state machine that you follow. 
\subsection{Writing the Draft}
Writing the draft was fairly easy using some of the techniques mentioned in class. Just sitting down and starting to write anything quickly evolved into writing actual sentences that were easy to convert to a real abstract. Starting an introduction was also easy using the technique of thinking like a child and writing a story. After writing a draft abstract and introduction transcribing the raw notes into a basic classification approach was fairly easy as a sub-bullet-ed list. The part that took longer was actually starting to classify papers and ensuring they fit the model neatly. I ended up going back and finding new papers on top of what I initially added. I realized a lot of the papers I thought fit my topic really didnt after trying to fit them to the model.

\end{document}
